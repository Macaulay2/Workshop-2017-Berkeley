\documentclass[11pt]{amsart}
\usepackage{calc,amssymb,amsthm,amsmath,fullpage    }
%\usepackage{mathtools}
\RequirePackage[dvipsnames,usenames]{xcolor}
\usepackage{hyperref}
\hypersetup{
bookmarks,
bookmarksdepth=3,
bookmarksopen,
bookmarksnumbered,
pdfstartview=FitH,
colorlinks,backref,hyperindex,
linkcolor=Sepia,
anchorcolor=BurntOrange,
citecolor=MidnightBlue,
citecolor=OliveGreen,
filecolor=BlueViolet,
menucolor=Yellow,
urlcolor=OliveGreen
}
\usepackage{alltt}
\usepackage{multicol}
%\usepackage{etex}
\usepackage{xspace}
\usepackage{rotating}
\interfootnotelinepenalty=100000

\usepackage{mabliautoref}
\usepackage{colonequals}
\frenchspacing
\input{kmacros3.sty}
\usepackage{stmaryrd}

\usepackage{verbatim}
\usepackage{enumerate}
\begin{document}
\title{{TestIdeals} package for \emph{Macaulay2}}
\author{Karl Schwede}
\date{\today}
\address{Department of Mathematics, University of Utah, 155 S 1400 E Room 233, Salt Lake City, UT, 84112}
\email{schwede@math.utah.edu}

\begin{abstract}
  This note describes a \emph{Macaulay2} package for computations in commutative rings prime related to $p^{-e}$-linear and Frobenius maps,
  singularities defined in terms of these maps,  various test ideals and modules, and ideals compatible with a given $p^{-e}$-linear map.
\end{abstract}


\subjclass[2010]{13A35}

\keywords{Macaulay2}

%\thanks{The first named author was supported in part by the NSF FRG Grant DMS \#1265261, NSF CAREER Grant DMS \#1252860 and a Sloan Fellowship.}
%\thanks{The second named author was supported in part by the NSF CAREER Grant DMS \#1252860.}
\maketitle

\section{Introduction}

This paper constructive methods for computing various objects related to commutative rings of prime characteristic $p$.
Such a ring $R$ comes equipped with a built-in endomorphism, namely the Frobenius endomorphism $f:R \rightarrow R$ which is the basis for many constructions and definitions
which affords a handle on many problems which is not otherwise available. Two notable examples of the use of the Frobenius endomorphism are the theory of tight closure {\hfill\large\color{red} [Add references]}\\
and the resulting theory of test ideals. {\hfill\large\color{red} [Add references]}\\

{\large\color{red} [Add history of the package]}

\subsection*{Acknowledgements}

We thank ??? for useful conversations and comments on the development of this package.

\section{Frobenius powers and Frobenius roots}

%%% Throughout  this paper $R$ will denote a polynomial ring over a field $\mathbb{K}$ of prime characteristic $p$.

Let $S$ denote any commutative ring of prime characteristic $p$.

\begin{definition}
For any ideal $I\subseteq S$ and any integer $e\geq I$, we define the \emph{$e$th Frobenius power of $I$} to be the ideal denoted $I^{[p^e]}$ which is
generated by all $p^e$th powers of elements in $I$.
\end{definition}

It is easy to see that, if $I$ is generated by $g_1, \dots, g_\ell$, $I^{[p^e]}$ is generated by $g_1^{p^e}, \dots, g_\ell^{p^e}$.


\begin{definition}
For any ideal $I\subseteq S$ and any integer $e\geq I$, we define the \emph{$e$th Frobenius root of $I$} to be the ideal denoted $I^{[1/p^{e}]}$ which is
the smallest ideal $J$ such that $I\subseteq J^{[p^e]}$, if such ideal exists.
\end{definition}

$e$th Frobenius roots exist in polynomial rings (cf.~\cite[\S 2]{BlickleMustataSmithDiscretenessAndRationalityOfFThresholds}) and in power series rings (cf.~\cite[\S 5]{KatzmanParameterTestIdealOfCMRings}


\begin{verbatim}
i2 :      R=ZZ/5[x,y,z]
i3 :      I=ideal(x^6*y*z+x^2*y^12*z^3+x*y*z^18)
                18    2 12 3    6
o3 = ideal(x*y*z   + x y  z  + x y*z)
o3 : Ideal of R
i4 :      frobeniusPower(1/5,I)
                2   3
o4 = ideal (x, y , z )
\end{verbatim}

We can extend the definition of Frobenius powers as follows
\begin{definition}{\hfill\large\color{red} [Add references]}\\
Let  $I\subseteq S$ be an ideal.
\begin{enumerate}
 \item[(a)] If $a$ is a positive integer with base-$p$ expansion  $a=a_0 + a_1 p +  \dots + a_r p^r$, we define
 $I^{[n]}=I^{a_0} \left(I^{a_1}\right)^{[p]} \dots  (I^{a_r})^{[p^r]}$. %\left( I^{a_r}\right)^{[p^r]}$ .
 \item[(b)] If $t$ is a non-negative rational number of the form $t = a/p^e$, we define  $I^{[t]} = (I^{[a]})^{[1/p^e]}.$
 \item[(c)] If $t$ is any non-negative rational number, and $\{a_n/p^{e_n}\}_{n\geq 1}$ is a sequence of rational numbers converging to $t$ from above, we define $I^{[t]}$
 to be the stable value of increasing chain of ideals $\{I^{[a_n/p^{e_n}]}\}_{n\geq 1}$.

\end{enumerate}
\end{definition}


\begin{verbatim}
i5 :      frobeniusPower(1/2, ideal(y^2-x^3))
o5 = ideal 1
o5 : Ideal of R
i6 :      frobeniusPower(5/6, ideal(y^2-x^3))
o6 = ideal (y, x)
o6 : Ideal of R
\end{verbatim}



\section{$p^{-e}$- and $p^{e}$-linear maps}

\section{$F$-singularities}

\section{Test ideals}

In this section, we explain how to compute parameter test modules, parameter test ideals, test ideals and HSLG-modules\footnote{Which can be used }.

\subsection{Parameter test modules}

Given an $F$-finite reduced ring $R$, the Frobenius map $R \to R^{1/p^e}$ is dual to $\omega_{R^{1/p^e}} \to \omega_R$.  As before


\subsection{Parameter test ideals}

\subsection{Test ideals}

\section{Ideals compatible with given $p^{-e}$-linear map}

\section{Future plans}

\bibliographystyle{skalpha}
\bibliography{MainBib}



\end{document}
